%%%%%%%%%%%%%%%%%%%%%%% file template.tex %%%%%%%%%%%%%%%%%%%%%%%%%
%
% This is a general template file for the LaTeX package SVJour3
% for Springer journals.          Springer Heidelberg 2010/09/16
%
% Copy it to a new file with a new name and use it as the basis
% for your article. Delete % signs as needed.
%
% This template includes a few options for different layouts and
% content for various journals. Please consult a previous issue of
% your journal as needed.
%
%%%%%%%%%%%%%%%%%%%%%%%%%%%%%%%%%%%%%%%%%%%%%%%%%%%%%%%%%%%%%%%%%%%
%
% First comes an example EPS file -- just ignore it and
% proceed on the \documentclass[•]{•} line
% your LaTeX will extract the file if required
\begin{filecontents*}{example.eps}
%!PS-Adobe-3.0 EPSF-3.0
%%BoundingBox: 19 19 221 221
%%CreationDate: Mon Sep 29 1997
%%Creator: programmed by hand (JK)
%%EndComments
gsave
newpath
  20 20 moveto
  20 220 lineto
  220 220 lineto
  220 20 lineto
closepath
2 setlinewidth
gsave
  .4 setgray fill
grestore
stroke
grestore
\end{filecontents*}
%
\RequirePackage{fix-cm}
%
%\documentclass{svjour3}                     % onecolumn (standard format)
%\documentclass[smallcondensed]{svjour3}     % onecolumn (ditto)
\documentclass[smallextended]{svjour3}       % onecolumn (second format)
%\documentclass[twocolumn]{svjour3}          % twocolumn
%
\smartqed  % flush right qed marks, e.g. at end of proof
%
% \usepackage{mathptmx}      % use Times fonts if available on your TeX system
%
% insert here the call for the packages your document requires
\usepackage{latexsym}
\usepackage{graphicx}
\usepackage{amsmath, amssymb, amsfonts, mathtools}
\usepackage{enumerate}
\usepackage{algorithm, algpseudocode}
\usepackage{txfonts, pxfonts}
\usepackage{grffile}
\usepackage{caption, subcaption}
\usepackage{listings}
\usepackage[table, xcdraw]{xcolor}
\usepackage{rotating}
\usepackage{multirow}
\usepackage{chronology}
\usetikzlibrary{arrows, shapes}
\usepackage{tabularx}
\usepackage{hyperref}
\usepackage{libertine}
\usepackage{pgfgantt}
\usepackage{lscape}
\usepackage{enumitem}

\usepackage{tikz} % drawing graph
\usetikzlibrary{arrows.meta} % drawing graph
%\usepackage[tight,footnotesize]{subfigure}

\input amssym.def
\input amssym.tex
%
% please place your own definitions here and don't use \def but
\newcommand{\mmlcpt}{$mbMML_{CPT}$ }
\newcommand{\mbptmml}{$MBPT_{mml}$ }
\DeclareMathOperator*{\argmin}{arg\,min}
\DeclarePairedDelimiter\floor{\lfloor}{\rfloor}
\newcommand{\ci}{\mathrel{\text{\scalebox{1.07}{$\perp\mkern-10mu\perp$}}}}
\newcommand{\independent}{\perp\mkern-9.5mu\perp}
\newcommand{\notindependent}{\centernot{\independent}}
\newcommand{\qedwhite}{\hfill \ensuremath{\Box}}
\algnewcommand\algorithmicforeach{\textbf{for each}}
\algdef{S}[FOR]{ForEach}[1]{\algorithmicforeach\ #1\ \algorithmicdo}

\makeatletter
% Taken from http://ctan.org/pkg/centernot
\newcommand*{\centernot}{%
  \mathpalette\@centernot
}
\def\@centernot#1#2{%
  \mathrel{%
    \rlap{%
      \settowidth\dimen@{$\m@th#1{#2}$}%
      \kern.5\dimen@
      \settowidth\dimen@{$\m@th#1=$}%
      \kern-.5\dimen@
      $\m@th#1\not$%
    }%
    {#2}%
  }%
}
\makeatother

%
% Insert the name of "your journal" with
% \journalname{myjournal}
%
% This file can be modified and used in other conferences as long
% as credit to the authors and supporting agencies is retained, this notice
% is not changed, and further modification or reuse is not restricted.

\begin{document}

\title{Logit and probit models
%On checking Markov blankets consistency with DAGs via graph immoralization%\thanks{Grants or other notes
%about the article that should go on the front page should be
%placed here. General acknowledgments should be placed at the end of the article.}
}
%\subtitle{Do you have a subtitle?\\ If so, write it here}

%\titlerunning{Short form of title}        % if too long for running head

\author{Kelvin Yang Li}

%\authorrunning{Short form of author list} % if too long for running head

\institute{Kelvin Yang Li \at
              Faculty of IT, Monash University, Victoria, Australia \\
              \email{yang.kelvinli@monash.edu}           %  \\
%             \emph{Present address:} of F. Author  %  if needed
}

\date{Started: 27 October 2018/ Finished: 27 October 2018}
% The correct dates will be entered by the editor

\maketitle

A \textit{sigmoid function} is a function having a S-shaped curve. It is monotonically increasing function that takes inputs between $(-\infty,\infty)$ and outputs between $(0,1$ or $(-1,1)$. Its mathematical form is 
\begin{align*}
S(x)=\frac{1}{1+e^{-x}}=\frac{e^x}{e^x+1}.
\end{align*}
Examples of sigmoid functions are the logistic function, hyperbolic tangent function, CDFs, etc. 

\section{Logit model}
The \textit{logistic distribution} is a continuous probability distribution that has bell shaped PDF curves. Its PDF is parameterized by mean $\mu$ and a scale parameter $s=q\sigma$ that is proportional to the standard deviation by $q=\sqrt{3}/\pi$. The PDF of the logistic distribution is expressed as 
\begin{align*}
f(x;\mu;s)=\frac{e^{-\frac{x-\mu}{s}}}{s(1+e^{-\frac{x-\mu}{s}})^2}=\frac{1}{s(e^{\frac{x-\mu}{2s}}+e^{-\frac{x-\mu}{2s}})^2}.
\end{align*}
The CDF of the logistic distribution is also known as the \textit{logistic function} and is expressed as 
\begin{align*}
F(x;\mu;s)=\frac{1}{1+e^{-\frac{x-\mu}{s}}},
\end{align*}
which is a function of the family of the logistic functions. The standard logistic function is when $\mu=0$ and $s=1$, so it has the same functional form as the sigmoid function. The logistic function often mentioned when introducing logit model refers to the standard logistic function (or the sigmoid function). 

In a logit model, each predictor $x_i$ has an associated parameter $\beta_i$, so the \textit{log-odds} $t=\beta_0 x_0+\cdots+\beta_m x_m = \vec{\beta}\cdot \vec{x} \in (-\infty,\infty)$ for $i\in [0,m]$, where $\beta_0$ is known as the intercept and setting $x_0=1$ is a trick for both programming convenience and expressing $t$ in a compact form. Since probability=odds/(1+odds), the logistic function converts the log-odds into probabilities by 
\begin{align*}
P(y=1\mid x) = \frac{1}{1+e^{-\vec{\beta}\cdot \vec{x}}}.
\end{align*}
The observed data for logit models are $x$s and and a binary $y$. We can assume there exists a latent variable $y^*=\vec{\beta}\cdot \vec{x}+\epsilon$, where $\epsilon$ is an unobserved error term but assumed to follow a standard logistic distribution. Hence, $y$ can be interpreted as an indicator of whether or not $y^*$ is positive. That is, if $\vec{\beta}\cdot \vec{x}+\epsilon >0$ then $y=1$, else $y=0$. Hence, we have 
\begin{align*}
P(y=1\mid x) &= p(\epsilon>-\vec{\beta}\cdot \vec{x}) \\
&= p(\epsilon<\vec{\beta}\cdot \vec{x}) \\
&= \frac{1}{1+e^{-\vec{\beta}\cdot \vec{x}}}.
\end{align*}

The logit model is a case of the family called \textit{generalized linear model}. It uses the standard logistic function as the link function. 

\section{Probit model}
The probit model is also in the family of generalized linear model, but uses a different link function from the logit model. In particular, the probit model uses the CDF $\Phi()$ of the standard normal distribution. The functional form of the probit model $P(y=1\mid x)=\Phi(\vec{beta}\cdot \vec{x})$ can also be derived using a latent variable $y^*$ and an unobserved error $\epsilon \sim N(0,1)$ that assumed to follow a standard normal distribution.

\end{document}