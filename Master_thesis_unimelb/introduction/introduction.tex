\chapter{Introduction}

This thesis studies various graph labelling problems over a collection of some interesting trees. 

In real world, when assigning channels to transmitters at different locations, certain rules are applied for a various of practical reasons. One of the major concern is o avoid interference when doing so. This requires large separation for two transmitters located close to each other. Bandwidth is a limited resource in real world. The larger a bandwidth is, the more money it will cost when setting up channels. The objective of channel assignment problem is to keep the bandwidth as small as possible while no interference occurs in a set of transmitters. 

Channel assignment problems were formulated by graph theorists into graph labelling problems in order to solve real world problems more accurate. This not only introduce an interesting area of research into mathematical world, but also saves resources, time and money in practice. In details, channel assignment problems can be represented by graph labelling problems, where vertices of a graph represent a set of transmitters and the non-negative integer labels for vertices represent the channel assigned to each of a set of transmitters. The aim of solving graph labelling problems becomes minimising the maximum used label. In this thesis, we will focus on two such problems, namely the linear and cyclic metric labelling problems. 

The linear metric labelling, normally denoted by $L(p_1, p_2, p_3, \dots)$, is an assignment $f: V(G) \rightarrow \{0, 1, 2, \dots, k\}$ of non-negative integers to vertices of a graph $G$ such that two vertices at distance $i$ apart must receive labels with separation at least $p_i$, where $p_i$ is the $i^{th}$ parameter of the $L(p_1, p_2, p_3, \dots)$-labelling. The difference between the maximum value and the minimum value of labels used is called the span of such a map. The aim of the $L(p_1, p_2, p_3, \dots)$-labelling problem is to find the smallest $k \in \mathbb{Z}_{\ge 0}$, which is called the minimum span of the $L(p_1, p_2, p_3, \dots)$-labelling of the graph $G$, which satisfies the above constraint. The $L(p_1, p_2)$-labelling problem has been widely studied. However, the $L(p_1,p_2,p_3)$-labelling problem is now attracting great interest because of its practical and theoretical interests. 

The cyclic metric labelling, denoted by $C(p_1, p_2, p_3, \dots)$, is similar to the $L(p_1, p_2, p_3, \dots)$-labelling. It is also an assignment $f: V(G) \rightarrow \{0, 1, 2, \dots, k\}$ of non-negative integers to vertices of a graph $G$. However, two vertices at distance $i$ apart now receive labels with separation at least $p_i \pmod{k}$. The aim of the $C(p_1, p_2, p_3, \dots)$-labelling problem is the same as that for the $L(p_1, p_2, p_3, \dots)$-labelling problem. That is, to find the smallest $k$ that satisfies the above constraint. Similarly, the span and the minimum span of the $C(p_1, p_2, p_3, \dots)$-labelling of the graph $G$ are both the same as the linear case. The $C(p_1, p_2, p_3, \dots)$-labelling problem has been as not broadly studied as the $L(p_1, p_2, p_3)$-labelling problem, however many good results about such a labelling have been obtained for various classes of graphs.  

We will mainly study the $L(p_1, p_2, p_3)$-labelling problem with $p_1 = h, p_2=p_3=1$. That is, the $L(h,1,1)$-labelling problem. Moreover, the collection of graphs that will be studied are the family of complete $m$-ary trees and its extensions. A complete $m$-ary tree is a tree $T$ such that every vertex of $T$, except its leaves, has exactly $m$ children. The reason why we choose to work on complete $m$-ary trees is because every tree is a subtree of a complete $m$-ary tree. Thus if we can find a tight upper bound for the family of complete $m$-ary trees, then this bound is also an upper bound for all trees. 

We begin by introducing some basic concepts in graph theory that will be used in this thesis. Chapter 3 focuses on the $L(h,1,1)$-labelling problem. We start off by giving some definitions and notations. Then we will review some known results of linear metric labelling problem for different types of graphs. Our results come in the following two sections. In Section 3.4, we prove the minimum span of the $L(h,1,1)$-labelling of complete $m$-ary trees to be $2m+h-1$ for height $2$ complete $m$-ary trees; $2m+h$ for height $3$ complete $m$-ary trees. Then in the next section, this result is extended to the $\lhpq$-labellings of complete $m$-ary trees. In Section 3.6, we further extended the $\lhpq$-labelling results to some larger families of trees, which contain the set of complete $m$-ary trees as a subset. 

In Chapter 4, we turn our attention to $C(h,1,1)$-labelling problem. The structure of Chapter 4 is the same as the structure of Chapter 3. We start off by introducing new definitions and notations. We then review some known results for cyclic metric labelling problems. Then in Section 4.3, we give our own results of the $C(h,1,1)$-labelling of complete $m$-ary trees as well as detailed proofs. Instead of an exact value, this time the results are two values, which depend on the number of children $m$ and the first parameter $h$. Precisely, the minimum span of the $C(h,1,1)$-labelling is $\max\{2m+h-1, 2h+m-1\}$ for height $2$ complete $m$-ary trees; $\max\{2m+h, 2h+m-1\}$ for height $3$ complete $m$-ary trees. In the next section, these results are extended to the $\chpq$-labelling of complete $m$-ary trees. We find a reasonable bound for complete $m$-ary trees with height $2$. For height $3$ complete $m$-ary trees, we only give an estimation of the upper bound. In Section 4.5, we again focus on those families of trees that we considered in Section 3.6. However, we only generalise the $C(h,1,1)$-labelling of these trees. 

Chapter 5 is a short conclusion of our results in this thesis. We also give a plan for some further studies in future. 

In the appendix, there is a table of notations that are widely used throughout this thesis. 








