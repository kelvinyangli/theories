\chapter*{\centering Abstract}
\addcontentsline{toc}{chapter}{Abstract}

The channel assignment problem is concerned with assigning a channel to each of a set of transmitters such that the maximum span of the channels is minimized under some distance constraints. Given $p_1, p_2, p_3, \dots$ all positive integers, an $L(p_1, p_2, p_3,\dots)$-labelling of a graph is an assignment of non-negative integers to its vertices such that vertices at distance $i$ apart must receive labels with separation at least $p_i$. The span of such a labelling is the difference between the maximum and minimum labels used. The objective is to find the minimum span over all $L(p_1, p_2, p_3, \dots)$-labellings of the graph. 

In this paper, we focus on $L(h,p,q)$-labellings of complete $m$-ary trees. We prove that the minimum span of the $\lhpq$-labelling of complete $m$-ary trees is an exact value. In the second part of this thesis, we introduce another labelling system the $C(p_1,p_2,p_3, \dots)$-labelling, called the cyclic metric labelling. We prove that the minimum span of the $C(h,1,1)$-labelling of complete $m$-ary trees is also an exact value, but this time it is dependent upon the parameter $m$ and $h$. 