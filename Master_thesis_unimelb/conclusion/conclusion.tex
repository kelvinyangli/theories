\chapter{Conclusion}

To finish this thesis, we provide a summary of all our major results. Moreover, at the end of this chapter, we will list some interesting topics for further studies. 

In both Chapter 3 and 4, we presented our results of the $L(h,1,1)$ and $C(h,1,1)$-labelling problems of complete $m$-ary trees. To achieve a more general result, we generalised our study to $\lhpq$-labelling problem of complete $m$-ary trees. The $C(h,1,1)$-labelling problem was also extended to the $\chpq$-labelling problem, though results have been achieved only for height $2$ complete $m$-ary trees. 

As every tree is a subtree of a complete $m$-ary tree, the upper bound for the set of complete $m$-ary trees is an upper bound for the set of all trees. By comparing $\lambda_{h,1,1}(\tmk)$ in this thesis with $\lambda_{h,1,1}(T)$ in \cite{zhou10}, we believe that our $\lambda_{h,1,1}(\tmk)$ value for $k \ge 3$ slightly improves the upper bound for $\lambda_{h,1,1}(T)$, for all trees $T$. 

There are not any known results for $C(h,1,1)$-labelling of trees. But we know that the $\t_{h,1,1}(\tmk)$ values presented in this thesis can also be used to bound above $\t_{h,1,1}(T)$ for all trees $T$, though this bound may not be optimal. 

Below we give a short list of some interesting topics in this area. 
\begin{enumerate}[(1)]
\item $L(h,1,1)$ and $\lhpq$-labelling of outerplanar graphs. 
\item $C(h,1,1)$ and $\chpq$-labelling of trees. In particular, check if $\t_{h,1,1}(\tmk)$ presented in this thesis is a tight upper bound for $\t_{h,1,1}(T)$, for all trees $T$.
\item $C(h,1,1)$ and $\chpq$-labelling of outerplanar graphs. 
\end{enumerate}